\documentclass[a11paper, 11pt]{article}

\usepackage{document}
\usepackage{titlepage}
\usepackage{float}
\usepackage{algorithm}
\usepackage{algpseudocode}
\usepackage{frpseudocode}
\usepackage[T1]{fontenc}
\usepackage[french]{babel}

% \addbibresource{bibliography.bib}
% \nofiles

% \institution{Université de Sherbrooke}
% \faculty{Faculté de génie}
% \department{Département de génie électrique et de génie informatique}
\title{Rapport d'APP}
\classnb{GEN241}
\class{Modélisation et programmation orientée objet}
\author{
  \addtolength{\tabcolsep}{-0.4em}
  \begin{tabular}{rcl} % Ajouter des auteurs au besoin
  Benjamin Chausse & -- & chab1704 \\
  Édouard Laurent  & -- & laue\textbf{XXXX} \\ % TODO: demander numéro du CIP
  \end{tabular}
}
\teacher{Eugène Morin}
% \location{Sherbrooke}
% \date{\today}


\begin{document}
\maketitle
\newpage
\tableofcontents
\newpage

\section{Diagrammes UML}

\section{Pseudo-code}

Ceci est un test pour voir comment le package frpseudocode fonctionne.
Du pseudo-code est pertinent pour l'APP ira ici.

\begin{algorithm}
\caption{Euclid's algorithm}\label{alg:euclid}
\begin{algorithmic}[1]
  \Function{Euclid}{$a, b$}
    \While{$a \neq b$}
      \If{$a > b$}
        \State $a \gets a - b$
      \Else
        \State $b \gets b - a$
      \EndIf
    \EndWhile
    \State \Return $a$
  \EndFunction
\end{algorithmic}
\end{algorithm}



% \newpage
% \printbibliography[heading=bibintoc]
\end{document}
